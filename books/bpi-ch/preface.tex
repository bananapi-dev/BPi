\section*{Audience}
\noindent{}This is an introduction to CPAM(Cross-Plaform Application Management) tool. 
Manage applications across different OS platform is a chellenage that system administrator need to face daily. 
You can learn the application management system (AMS) that come with each OS or you can learn a CPAM tool to interact with those local AMS without the need to master each and indivisual AMS programs.


The intended audience of this book is people who need to package software before installing software into an OS system.
A list of local package manager has been introduced in one chapter depth and hyper package manager tools like TWW tool sets are presented in later chapters.

A Hello World type of simple package creation will be used throughout this book to illustrate the differences from using various package managers.
\section*{Book layout}
\begin{description}
\item[Chapter \ref{chap:preface}: \titleref{chap:preface}]
This chapter itself.
\item[Chapter \ref{chap:intro}: \titleref{chap:intro}]
Software does need to be updated,deleted and queryed after it got installed into prodcution Operating System. 
This chapter try to describe the flow of a software from birth to death.
\item[Chapter \ref{chap:autotools}: \titleref{chap:autotools}]
\item[Chapter \ref{chap:rpm}: \titleref{chap:rpm}]
RPM package manager got created to manage RedHat's Linux distribution. It has been used by other Linux distributions also.
\item[Chapter \ref{chap:pkgadd}: \titleref{chap:pkgadd}]
Solaris 10 and prior use SVR4 pkgadd package manager to manage software.
\item[Chapter \ref{chap:ips}: \titleref{chap:ips}]
Starting Solaris 11, a new package management system created to correct pkgadd's shortcoming and catch up with modern package manager.
\item[Chapter \ref{chap:deb}: \titleref{chap:deb}]
Debean P.M. is used to manager software in Debian and Ubuntu Linux distributions.
\item[Chapter \ref{chap:msi}: \titleref{chap:msi}]
MSI is called to refered to a collection of tool sets in Windows OS platform to manage software.
\item[Chapter \ref{chap:pkg}: \titleref{chap:pkg}]
pkg is referring to tool like PackageMaker in OSX to create software .pkg type of packages.
\item[Chapter \ref{chap:swdepot}: \titleref{chap:swdepot}]
HP-UX use swdeopt system to create .depot type of package.
\item[Chapter \ref{chap:sbutils}: \titleref{chap:sbutils}]
Describes sbutils toolsets which consiste of sb to automate the software creation porcess across different operation system. 
The software creation process of download,unpack,configure,compile and install the binary.

\item[Chapter \ref{chap:pbutils}: \titleref{chap:pbutils}] 
Describes pbutils toolsets which consiste of pb to create software packages by calling native  package manager across different operation system.
\item[Chapter \ref{chap:pkgutils}: \titleref{chap:pkgutils}] 
Describes pkgutils toolsets which consiste of pkg-inst,pkg-rm and pkg-info to manage software package across different operation system.

\end{description}

I hope you will enjoy this book and the language Go.

\section*{Settings used in this book}
\begin{itemize}
\item Go itself is installed in \file{\~{}/go}, and \var{\$GOROOT} is set to \var{GOROOT=\~{}/go} ;
\item Go source code we want to compile ourself is placed in \file{\~{}/g/src} and
\var{\$GOPATH} is set to \var{GOPATH=\~{}/g} .
\end{itemize}




