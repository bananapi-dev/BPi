

\noindent{} \index{GCE} has REST API interface to be interact with.

\begin{description}
\item[gcutil in Python]
Go strives to keep things small and beautiful. You should
be able to do a lot in only a few lines of code; \index{xelatex} 
\item[gcelib in Python ]
\item[gcelib in Golang] 

\end{description}

\section{gcutil}
\label{sec:gcutil}

\section{gsutil}
\label{sec:gsutil}
The preferred way to build a Go program is to use the \prog{go} tool.\index{tooling!go}
To build \prog{helloworld} we just enter:
\begin{display}
\pr \user{go build helloworld.go}
\end{display}
\index{gce toolsets!gsutil}
This results in an executable called \prog{helloworld}.

